\subsection{%
  Лекция \texttt{23.11.07}.%
}

\subheader{Сходимость случайных величин}

\subsubheader{I.}{Сходимость \quote{почти наверное}}

\begin{definition}
  Случайная величина \(\xi\) имеет свойство \(Cnd\) \quote{почти наверное}, если
  \(\prob{\xi \text{ имеет свойство } Cnd} = 1\). Иными словами \(\prob{\omega
  \in \Omega \given \xi(\omega) \text{ не имеет свойства } Cnd} = 0\).
\end{definition}

\begin{definition}
  Последовательность случайных величин \(\seq{\xi_n}\) \quote{почти наверное}
  сходится к случайной величине \(\xi\), если

  \begin{equation*}
    \prob{\omega \in \Omega \given \xi_n (\omega) \to \xi(\omega)} = 1
  \end{equation*}

  Обозначается \(\xi_n \Rarr{\text{п.н.}} \xi\).
\end{definition}

\subsubheader{II.}{Сходимость по вероятности}

\begin{definition}
  Последовательность случайных величин \(\seq{\xi_n}\) сходится по вероятности к
  случайной величине \(\xi\), если

  \begin{equation*}
    \forall \epsilon > 0 \given
    \prob{\abs{\xi_n - \xi} < \epsilon} \to 1
    \qquad (n \to \infty)
  \end{equation*}

  Обозначается \(\xi_n \Rarr{\probP} \xi\).
\end{definition}

\begin{remark}
  Это аналог сходимости по мере, только по вероятностной мере. Можно понимать
  как ослабленную равномерную сходимость.

  \begin{equation*}
    \prob{
      \omega \in \Omega
      \given \xi_n (\omega) \text{ не укладывается в } \epsilon\text{-полосу}
    } \to 0
  \end{equation*}
\end{remark}

\begin{remark}
  В общем случае

  \begin{equation*}
    \xi_n \Rarr{\probP} \xi
    \not\implies
    \expected{\xi_n} \Rarr{n \to \infty} \expected{\xi}
  \end{equation*}
\end{remark}

\begin{theorem}
  Пусть \(\forall n \colon \abs{\xi_n} \le C = const\), тогда

  \begin{equation*}
    \xi_n \Rarr{\probP} \xi
    \implies
    \expected{\xi_n} \Rarr{n \to \infty} \expected{\xi}
  \end{equation*}
\end{theorem}

\subsubheader{III}{Слабая сходимость (сходимость по функции распределения)}

\begin{definition}
  Последовательность случайных величин \(\xi_n\) слабо сходится к случайной
  величине \(\xi\), если

  \begin{equation*}
    F_{\xi_n} (x) \Rarr{n \to \infty} F_{\xi}
    \qquad \forall x \given F_{\xi} (x) \text{ непрерывна в точке } x
  \end{equation*}

  Таким образом, слабая сходимость это сходимость функций распределений во всех
  точках непрерывности. Обозначается \(\xi_n \rightrightarrows \xi\).
\end{definition}

\subheader{Связь между видами сходимости}

\begin{theorem}
  \begin{equation*}
    \xi_n \Rarr{\text{п.н.}} \xi
    \implies
    \xi_n \Rarr{\probP} \xi
    \implies
    \xi_n \rightrightarrows \xi
  \end{equation*}
\end{theorem}

\begin{theorem} \label{thr:weak-conv-to-const}
  Если \(\xi_n \rightrightarrows C = const\), то \(\xi_n \Rarr{\probP} C\).
\end{theorem}

\begin{proof}
  \begin{equation*}
    \begin{aligned}
      \xi_n \rightrightarrows C = const
      \implies
      F_{\xi_n} (x) \to F_C (x) = \begin{cases}
        0, & x < C \\
        1, & x \ge C
      \end{cases}
      \text{ за исключением точки разрыва } x = C
    \\
      \prob{\abs{\xi_n - C} < \epsilon}
      = \prob{C - \epsilon < \xi_n < C + \epsilon}
      \ge \prob{C - \frac{\epsilon}{2} \le \xi_n < C + \epsilon}
      = F_{\xi_n} (C + \epsilon) - F_{\xi_n} \prh{C - \frac{\epsilon}{2}}
    \\
      F_{\xi_n} (C + \epsilon) - F_{\xi_n} \prh{C - \frac{\epsilon}{2}}
      \Rarr{n \to \infty}
      F_C (C + \epsilon) - F_C \prh{C - \frac{\epsilon}{2}}
      = 1 - 0
      = 1
    \end{aligned}
  \end{equation*}

  Т.к. \(\prob{\abs{\xi_n - C} < \epsilon} \le 1\), то по теореме о двух
  милиционерах \(\prob{\abs{\xi_n - C} < \epsilon} \Rarr{n \to \infty} 1\).
\end{proof}

\begin{remark}
  В общем случае из слабой сходимости не только не следует сходимость по
  вероятности, но даже бессмысленно говорить об этом, т.к. слабая сходимость ---
  это, по существу, не сходимость случайных величин, а сходимость их распределений.
\end{remark}

\begin{example}
  Пусть \(\xi \in N(0; 1)\), тогда \(\eta = -\xi \in N(0; 1)\). Получаем, что
  \(F_{\xi} (x) = F_{\eta} (x)\). Ясно, что

  \begin{equation*}
    \xi_n \rightrightarrows \xi
    \not\implies
    \xi_n \Rarr{\probP} \xi
  \end{equation*}

  Т.к. в противном случае \(\xi_n \Rarr{\probP} \eta = -\xi\).
\end{example}

\subheader{Ключевые неравенства}

\subsubheader{I.}{Неравенство Маркова}

\begin{theorem} \label{thr:markov-inequality}
  \begin{equation*}
    \forall \epsilon > 0 \given
    \prob{\abs{\xi} \ge \epsilon} \le \frac{\expected{\abs{\xi}}}{\epsilon}
  \end{equation*}
\end{theorem}

\begin{proof}
  \begin{equation*}
    I_A (\omega) = \begin{cases}
      0, \omega \notin A \text{ событие } A \text{ не происходит} \\
      1, \omega \in A \text{ событие } A \text{ происходит}
    \end{cases}
  \end{equation*}

  \(I_A (\omega)\) имеет распределение Бернулли с параметром \(p\), где \(p =
  \prob{A} \) и \(\expected{I_A (\omega)} = p = \prob{A}\).

  \begin{equation*}
    \abs{\xi}
    \ge \abs{\xi} \cdot I \prh{\abs{\xi} \ge \epsilon}
    \ge \epsilon \cdot I \prh{\abs{\xi} \ge \epsilon}
  \end{equation*}

  Применим математическое ожидание к обеим частям неравенства.

  \begin{equation*}
    \expected{\abs{\xi}}
    \ge \expected{\epsilon \cdot I \prh{\abs{\xi} \ge \epsilon}}
    = \epsilon \cdot \expected{I \prh{\abs{\xi} \ge \epsilon}}
    = \epsilon \cdot \prob{\abs{\xi} \ge \epsilon}
    \implies
    \prob{\abs{\xi} \ge \epsilon}
    \le \frac{\expected{\abs{\xi}}}{\epsilon}
  \end{equation*}
\end{proof}

\subsubheader{II.}{Неравенство Чебышева}

\begin{theorem} \label{thr:chebyshev-inequality}
  \begin{equation*}
    \prob{\abs{\xi - \expected{\xi}} \ge \epsilon}
    \le
    \frac{\variance{\xi}}{\epsilon^2}
  \end{equation*}
\end{theorem}

\begin{proof}
  По \ref{thr:markov-inequality} имеем

  \begin{equation*}
    \prob{\abs{\xi - \expected{\xi}} \ge \epsilon}
    = \prob{\abs{\xi - \expected{\xi}}^2 \ge \epsilon^2}
    \le \frac{\expected{\prh{\xi - \expected{\xi}}^2}}{\epsilon^2}
    = \frac{\variance{\xi}}{\epsilon^2}
  \end{equation*}
\end{proof}

\begin{lemma}{Оценка для стандартизованной случайной величины}
  Пусть \(\eta = \frac{\xi - \expected{\xi}}{\stder{\xi}}\), тогда

  \begin{equation*}
    \prob{\abs{\eta} \ge \epsilon} \le \frac{1}{\epsilon^2}
  \end{equation*}
\end{lemma}

\subsubheader{III.}{Правило трех сигм}

\begin{theorem}
  \begin{equation*}
    \prob{\abs{\xi - \expected{\xi}} \ge 3 \stder{\xi}} \le \frac{1}{9}
  \end{equation*}
\end{theorem}

\begin{proof}
  По \ref{thr:chebyshev-inequality} имеем

  \begin{equation*}
    \prob{\abs{\xi - \expected{\xi}} \ge 3 \stder{\xi}}
    \le \frac{\variance{\xi}}{(3 \stder{\xi})^2}
    = \frac{\variance{\xi}}{9 \stder{\xi}^2}
    = \frac{1}{9}
  \end{equation*}
\end{proof}

\subheader{Среднее арифметическое случайных величин}

Пусть \(\xi_1, \dotsc, \xi_n\)~--- независимая, одинаково распределенная
случайная величина с конечным вторым моментом. Обозначим \(a =
\expected{\xi_i}\), \(d = \variance{\xi_i}\), \(\sigma = \stder{\xi_i}\), \(S_n
= \xi_1 + \dotsc + \xi_n\). Тогда

\begin{equation*}
  \expected{\frac{S_n}{n}}
  = \expected{\frac{\xi_1 + \dotsc + \xi_n}{n}}
  = \frac{1}{n} \prh{\expected{\xi_1} + \dotsc + \expected{\xi_n}}
  = \frac{1}{n} \cdot a n
  = a
\end{equation*}

Математическое ожидание осталось прежним.

\begin{equation*}
  \variance{\frac{S_n}{n}}
  = \variance{\frac{\xi_1 + \dotsc + \xi_n}{n}}
  = \frac{1}{n^2} \prh{\variance{\xi_1} + \dotsc + \variance{\xi_n}}
  = \frac{1}{n^2} \cdot d n
  = \frac{d}{n}
\end{equation*}

Дисперсия уменьшилась в \(n\) раз.

\begin{equation*}
  \stder{\frac{S_n}{n}}
  = \frac{\sigma}{\sqrt{n}}
\end{equation*}

Среднеквадратическое отклонение уменьшилось в \(\sqrt{n}\) раз.

\subheader{Законы больших чисел}

\subsubheader{I.}{Закон больших чисел Чебышева}

\begin{theorem} \label{thr:chebyshev-big-num-law}
  Пусть \(\xi_1, \dotsc, \xi_n, \dotsc\)~--- последовательность независимых
  одинаково распределенных случайных величин с конечным вторым моментом, тогда

  \begin{equation*}
    \frac{\xi_1 + \dotsc + \xi_n}{n} \Rarr{\probP} \expected{\xi_1}
  \end{equation*}
\end{theorem}

\begin{proof}
  Обозначим \(a = \expected{\xi_i}\), \(d = \variance{\xi_i} < \infty\), \(S_n =
  \sum_{i = 1}^{n} \xi_i\). Тогда по \ref{thr:chebyshev-inequality} имеем

  \begin{equation*}
    \prob{\abs{\frac{S_n}{n} - a} \ge \epsilon}
    = \prob{\abs{\frac{S_n}{n} - \expected{\frac{S_n}{n}}} \ge \epsilon}
    \le \frac{\variance{\frac{S_n}{n}}}{\epsilon^2}
    = \under{\frac{d}{\epsilon^2}}{= const} \cdot \under{\frac{1}{n}}{\to 0}
    \Rarr{n \to \infty} 0
    \implies \prob{\abs{\frac{S_n}{n} - a} \le \epsilon}
    \Rarr{n \to \infty} 1
  \end{equation*}

  что и означает сходимость по вероятности.
\end{proof}

Среднее арифметическое большого числа независимых одинаковых случайных величин
\quote{стабилизируется} около математического ожидания. Или при больших \(n\)
случайность переходит в почти закономерность.

Статистический смысл: при большом объеме \(n\) экспериментальных данных их
среднее арифметическое даст достаточно точную оценку неизвестного
математического ожидания.

\begin{remark}
  Попутно получили удобную, но грубую оценку:

  \begin{equation*}
    \prob{\abs{\frac{S_n}{n} - a} \ge \epsilon}
    \le \frac{\variance{\xi_1}}{n \epsilon^2}
  \end{equation*}
\end{remark}

\subsubheader{II.}{Закон больших чисел Бернулли}

\begin{theorem}
  Пусть \(v_n\)~--- число появления события \(A\) в серии из \(n\) независимых
  испытаний, \(p_n = \prob{A}\)~--- вероятность появления \(A\) при одном
  испытании. Тогда

  \begin{equation*}
    \frac{v_n}{n} \Rarr{\probP} \prob{A}
    \qquad
    \prob{\abs{\frac{v_n}{n} - p} \ge \epsilon}
    \le \frac{p (1 - p)}{n \epsilon^2}
    \qquad
    (\forall \epsilon > 0)
  \end{equation*}
\end{theorem}

\begin{proof}
  \(v_n = \xi_1 + \dotsc + \xi_n\), где \(\xi_i\)~--- число появления события
  \(A\) при \(i\)-том испытании, причем \(\xi_i\) независимые и имеют одинаковое
  распределение \(B_p\). Т.к. \(\expected{\xi_i} = p\), \(\variance{\xi_i} = p
  q\), то по \ref{thr:chebyshev-big-num-law}

  \begin{equation*}
    \begin{aligned}
      \frac{v_n}{n} = \frac{S_n}{n}
      \Rarr{\probP}
      \expected{\xi_1} = \prob{A}
    \\
      \prob{\abs{\frac{v_n}{n} - p} \ge \epsilon}
      \le \frac{p q}{n \epsilon^2}
    \end{aligned}
  \end{equation*}
\end{proof}

\subsubheader{III.}{Закон больших чисел Хинчина}

\begin{theorem} \label{thr:khinchin-big-num-law}
  Пусть \(\xi_1, \dotsc, \xi_n, \dotsc\) последовательность независимых
  одинаково распределенных случайных величин с конечным первым моментом
  \(\expected{\xi_1} < \infty\), тогда

  \begin{equation*}
    \frac{\xi_1 + \dotsc + \xi_n}{n}
    \Rarr{\probP}
    \expected{\xi_1}
  \end{equation*}
\end{theorem}

\subsubheader{IV.}{Усиленный закон больших чисел Колмогорова}

\begin{theorem}
  В условиях теоремы \ref{thr:khinchin-big-num-law}

  \begin{equation*}
    \frac{\xi_1 + \dotsc + \xi_n}{n}
    \Rarr{\text{п.н.}}
    \expected{\xi_1}
  \end{equation*}
\end{theorem}

\subsubheader{V.}{Закон больших чисел Маркова}

\begin{theorem}
  Пусть имеется последовательность случайных величин \(\xi_1, \dotsc, \xi_n,
  \dotsc\) таких, что \(\variance{S_n} = \smallo(n^2)\), и их вторые моменты
  конечны. Тогда

  \begin{equation*}
    \frac{S_n}{n}
    \Rarr{\probP}
    \frac{\expected{\xi_1} + \dotsc + \expected{\xi_n}}{n}
  \end{equation*}
\end{theorem}

\begin{proof}
  \begin{equation*}
    \expected{\frac{S_n}{n}}
    = \frac{\expected{\xi_1} + \dotsc + \expected{\xi_n}}{n}
  \end{equation*}

  И по \ref{thr:chebyshev-inequality} получаем

  \begin{equation*}
    \prob{\abs{\frac{S_n}{n} - \expected{\frac{S_n}{n}}} \ge \epsilon}
    \le \frac{\variance{\frac{S_n}{n}}}{\epsilon^2}
    = \frac{\variance{S_n}}{n^2 \epsilon^2}
    = \frac{\smallo(n^2)}{n^2 \epsilon^2}
    = \under{\frac{\smallo(n^2)}{n^2}}{\to 0}
      \cdot \under{\frac{1}{\epsilon^2}}{= const}
    \Rarr{n \to \infty} 0
  \end{equation*}
\end{proof}

\subheader{Центральная предельная теорема}

\begin{theorem}[Центральная предельная теорема Ляпунова]
  Пусть \(\xi_1, \dotsc, \xi_n\)~--- последовательность независимых одинаково
  распределенных случайных величин с конечным вторым моментом \(\variance{\xi_1}
  < \infty\) и \(S_n = \sum_{i = 1}^{n} \xi_i\). Тогда имеет место слабая
  сходимость

  \begin{equation*}
    \frac{S_n - n \expected{\xi_1}}{\sqrt{n \variance{\xi_1}}}
    \rightrightarrows N(0; 1)
  \end{equation*}

  Т.е. стандартизованная сумма в пределе имеет стандартное нормальное
  распределение.
\end{theorem}

\begin{remark}
  Теорему можно сформулировать по-другому. Пусть \(a = \expected{\xi_1}\),
  \(\sigma = \stder{\xi_1}\). Тогда, как уже было показано
  \(\expected{\frac{S_n}{n}} = a\) и \(\variance{\frac{S_n}{n}} =
  \frac{\sigma}{\sqrt{n}}\), при этом

  \begin{equation*}
    \frac{\frac{S_n}{n} - \expected{\frac{S_n}{n}}}{\stder{\frac{S_n}{n}}}
    \rightrightarrows N(0; 1)
  \end{equation*}

  Т.е. стандартизованное среднее арифметическое в пределе имеет стандартное
  нормальное распределение.
\end{remark}

\begin{remark}
  В грубой формулировке теорема будет иметь вид

  \begin{equation*}
    \frac{S_n}{n} \rightrightarrows N \prh{a; \frac{\sigma^2}{n}}
  \end{equation*}
\end{remark}
