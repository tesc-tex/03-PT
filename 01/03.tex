\subsection{%
  Лекция \texttt{23.09.19}.%
}

Если два события не являются независимыми и известно, что одно из них произошло,
то на основе этой информации мы можем уточнить вероятность наступления другого.
Вероятность данного в таком случае называется условной.

Условная вероятность \(\condprob{A}{B}\) это вероятность события \(A\)
вычисленная в предположении, что событие \(B\) уже произошло.

\begin{remark}
  Существует другой обозначение \(\probP_B (A)\), но мы не будем его
  использовать.
\end{remark}

\begin{example}
  Подбросили кубик. Известно, что выпало больше трех очков. Какова вероятность
  того, что выпало четное число очков?

  \solution{} пусть \(A\)~--- \quote{выпало четное число очков}, а \(B\)~---
  \quote{выпало более трех очков}. Тогда

  \begin{equation*}
    \prob{A} = \frac{3}{6} = \frac{1}{2}
    \qquad
    \prob{B} = \frac{3}{6} = \frac{1}{2}
  \end{equation*}

  Тогда получаем, что

  \begin{equation*}
    \condprob{A}{B}
    = \frac{2}{3}
    = \frac{\frac{2}{6}}{\frac{3}{6}}
    = \frac{\prob{A B}}{\prob{B}}
  \end{equation*}

  Такое же соотношение можно получить с помощью геометрической вероятности
  (\figref{01_03_01}).

  \begin{equation*}
    \condprob{A}{B}
    = \frac{S_{AB}}{S_B}
    = \frac{\frac{S_{AB}}{S_{\Omega}}}{\frac{S_B}{S_{\Omega}}}
    = \frac{\prob{A B}}{\prob{B}}
  \end{equation*}
\end{example}

\galleryone{01_03_01}{Геометрический подход к условной вероятности}

\begin{remark}
  Вывести эту формулу из аксиом вероятности нельзя, поэтому она берется в
  качестве определения.
\end{remark}

\begin{definition} \label{def:cond-prob}
  Условной вероятностью события \(A\) при условии, что имело место событие \(B\)
  называется величина

  \begin{equation*}
    \condprob{A}{B} = \frac{\prob{A B}}{\prob{B}}
  \end{equation*}
\end{definition}

\subheader{Формула произведения вероятностей}

Из определения условной вероятности (\ref{def:cond-prob}) следует, что

\begin{equation*}
  \prob{A B}
  = \prob{B} \condprob{A}{B}
  = \prob{A} \condprob{B}{A}
\end{equation*}

Или, в общем случае, имеем

\begin{theorem}[Формула произведения вероятностей] \label{thr:prob-mult}
  \begin{equation*}
    \prob{A_1 \dotsc A_n}
    = \prob{A_1}
      \condprob{A_2}{A_1}
      \condprob{A_3}{A_1 A_2}
      \dotsc
      \condprob{A_n}{A_1 \dotsc A_{n - 1}}
  \end{equation*}
\end{theorem}

\begin{proof}
  Мат. индукция.

  \textbf{База:} \(n = 2\).
  
  Этот случай уже рассмотрен ранее, он следует из определения условной
  вероятности.

  \textbf{Переход:} \(n - 1 \to n\)

  \begin{equation*}
    \begin{aligned}
      \prob{\under{A_1 \dotsc A_{n - 1}}{} A_n}
      & = \prob{A_1 \dotsc A_{n - 1}}
        \condprob{A_n}{A_1 \dotsc A_{n - 1}}
    \\
      & = \under{
          \prob{A_1}
          \condprob{A_2}{A_1}
          \condprob{A_3}{A_1 A_2}
          \dotsc
          \condprob{A_{n - 1}}{A_1 \dotsc A_{n - 2}}
        }{
          \prob{A_1 \dotsc A_{n - 1}}
        }
        \condprob{A_n}{A_1 \dotsc A_{n - 1}}
    \end{aligned}
  \end{equation*}
\end{proof}

\begin{remark}
  Условная вероятность определена при \(\prob{B} \neq 0\), поэтому формула
  умножения (\ref{thr:prob-mult}) верна при \(\prob{A_1 \dotsc A_{n - 1}} \neq
  0\).
\end{remark}

\begin{lemma}
  Покажем, что равносильны два определения независимости: \(\prob{A B} =
  \prob{A} \prob{B}\) и \(\condprob{A}{B} = \prob{A}\)
\end{lemma}

\begin{proof}
  \begin{equation*}
    \condprob{A}{B}
    = \frac{\prob{A B}}{\prob{B}}
    =  \prob{A}
    \iff
    \prob{A B} = \prob{A} \prob{B}
  \end{equation*}
\end{proof}

\begin{example}
  В коробке \(3\) красных карандаша и \(2\) синих. Вынули три карандаша. Найти
  вероятность того, что первые два красные, а третий~--- синий.

  \solution{} пусть \(A_1\)~--- \quote{первый красный}, \(A_2\)~---
  \quote{второй красный} и \(A_3\)~--- \quote{третий синий}.

  \begin{equation*}
    \prob{A_1 A_2 A_3}
    = \prob{A_1} \condprob{A_2}{A_1} \condprob{A_3}{A_1 A_2}
    = \frac{3}{5} \cdot \frac{2}{4} \cdot \frac{2}{3}
    = \frac{1}{5}
  \end{equation*}
\end{example}

\subheader{Полная группа событий}

\galleryone{01_03_02}{Иллюстрация полной группы событий}

\begin{definition}
  События \(H_1, \dotsc, H_n\) образуют полную группу событий, если они попарно
  несовместны и содержат все элементарные исходы.

  \begin{equation*}
    \begin{cases}
      \forall i \neq j \given H_i H_j = \varnothing \\
      H_1 + \dotsc + H_n = \Omega
    \end{cases}
  \end{equation*}
\end{definition}

\begin{corollary}
  \begin{equation*}
    \prob{H_1} + \dotsc + \prob{H_n} = 1
  \end{equation*}
\end{corollary}

\begin{proof}
  Т.к. события \(H_1, \dotsc, H_n\) несовместны, то вероятность их суммы равна
  сумме их вероятностей, и при этом она равна единице, т.к. эти события содержат
  все элементарные исходы.
\end{proof}

\begin{remark}
  События \(H_i\) еще называются гипотезами.
\end{remark}

\subheader{Формула полной вероятности}

\begin{theorem}
  Пусть \(H_1, \dotsc, H_n, \dotsc\)~--- полная группа событий. Тогда

  \begin{equation*}
    \prob{A} = \sum_{i = 1}^{\infty} \prob{H_i} \condprob{A}{H_i}
  \end{equation*}
\end{theorem}

\begin{proof}
  \begin{equation*}
    \prob{A}
    = \prob{\Omega A}
    = \prob{\prh{H_1 + \dotsc + H_n + \dotsc} A}
    = \prob{H_1 A + \dotsc + H_n A + \dotsc}
  \end{equation*}

  Заметим, что т.к. события \(H_i\) из полной группы, то они несовместны, значит
  и полученные произведения \(H_i A\) также несовместны. Применяя формулу
  произведения вероятностей имеем

  \begin{equation*}
    \prob{A}
    = \sum_{i = 1}^{\infty} \prob{H_i A}
    = \sum_{i = 1}^{\infty} \prob{H_i} \condprob{A}{H_i}
  \end{equation*}
\end{proof}

\subheader{Формула Байеса (формула проверки гипотез)}

\begin{theorem}
  Пусть \(H_1, \dotsc, H_n\)~--- полная группа событий. Известно, что событие
  \(A\) произошло. Тогда

  \begin{equation*}
    \condprob{H_k}{A}
    = \frac{\prob{H_k} \condprob{A}{H_k}}
      {\sum_{i = 1}^{\infty} \prob{H_i} \condprob{A}{H_i}}
  \end{equation*}
\end{theorem}

\begin{proof}
  Раскрываем условную вероятность по определению, после чего в числителе
  применяем формулу произведения вероятностей, а в знаменателе~--- формулу
  полной вероятности.

  \begin{equation*}
    \condprob{H_k}{A}
    = \frac{\prob{H_k A}}{\prob{A}}
    = \frac{\prob{H_k} \condprob{A}{H_k}}
      {\sum_{i = 1}^{\infty} \prob{H_i} \condprob{A}{H_i}}
  \end{equation*}
\end{proof}

\begin{example}
  В первой коробке \(4\) белых и \(2\) черных шара, а во второй~--- \(1\) белый
  и \(2\) черных. Из первой коробки во вторую переложили два шара. Затем из
  второй коробки достали шар. Какова вероятность того, что он оказался белым?

  \solution{} пусть \(H_1\)~--- во вторую коробку переложили \(2\) белых шара,
  \(H_2\)~--- переложили \(2\) черных шара, \(H_3\)~--- переложили \(2\) шара
  разного цвета. \(A\)~--- из второй коробки достали белый шар, тогда

  \begin{equation*}
    \begin{array}{lll}
      \prob{H_1} = \frac{4}{6} \cdot \frac{3}{5}= \frac{6}{15}
      &
      \prob{H_2} = \frac{2}{6} \cdot \frac{1}{5} = \frac{1}{15}
      &
      \prob{H_3}
        = \frac{4}{6} \cdot \frac{2}{5} + \frac{2}{6} \cdot \frac{4}{5}
        = \frac{8}{15}
    \\ \\
      \condprob{A}{H_1} = \frac{3}{5}
      &
      \condprob{A}{H_2} = \frac{1}{5}
      &
      \condprob{A}{H_3} = \frac{2}{5}
    \end{array}
  \end{equation*}

  По формуле полной вероятности получаем

  \begin{equation*}
    \prob{A}
    = \prob{H_1} \condprob{A}{H_1}
      + \prob{H_2} \condprob{A}{H_2}
      + \prob{H_3} \condprob{A}{H_3}
    = \frac{6}{15} \cdot \frac{3}{5}
      + \frac{1}{15} \cdot \frac{1}{5}
      + \frac{8}{15} \cdot \frac{2}{5}
    = \frac{18 + 1 + 16}{75}
    = \frac{7}{15}
  \end{equation*}
\end{example}

\begin{example} \label{ex:Bayes}
  По статистике раком болен \(1\)\% населения. Тест дает правильный результат в 
  \(99\)\% случаев. Анализ оказался положительным. Какова вероятность того, что
  человек болен?

  \solution{} пусть \(H_1\)~--- человек болен, а \(H_2\)~--- здоров. Событие
  \(A\)~--- анализ положительный. Тогда

  \begin{equation*}
    \begin{array}{ll}
      \prob{H_1} = 0.01
      &
      \prob{H_2} = 0.99
    \\
      \condprob{A}{H_1} = 0.99
      &
      \condprob{A}{H_2} = 0.01
    \end{array}
  \end{equation*}

  Применяем формулу Байеса.

  \begin{equation*}
    \condprob{H_1}{A}
    = \frac{\prob{H_1} \condprob{A}{H_1}}
      {\prob{H_1} \condprob{A}{H_1} + \prob{H_2} \condprob{A}{H_2}}
    = \frac{0.01 \cdot 0.99}{0.01 \cdot 0.99 + 0.99 \cdot 0.01}
    = \frac{1}{2}
  \end{equation*}
\end{example}

\begin{example}
  Пусть после ситуации, описанной в \ref{ex:Bayes}, провели второй
  \textbf{независимый} анализ, и он оказался положительным. Какова вероятность
  того, что человек болен?

  \solution{} используя обозначения из предыдущего примера получаем

  \begin{equation*}
    \begin{array}{ll}
      \prob{H_1} = 0.01
      &
      \prob{H_2} = 0.99
    \\
      \condprob{A A}{H_1} = 0.99 \cdot 0.99
      &
      \condprob{A A}{H_2} = 0.01 \cdot 0.01
    \end{array}
  \end{equation*}

  Применяем формулу Байеса.

  \begin{equation*}
    \condprob{H_1}{A A}
    = \frac{\prob{H_1} \condprob{A A}{H_1}}
      {\prob{H_1} \condprob{A A}{H_1} + \prob{H_2} \condprob{A A}{H_2}}
    = \frac{0.01 \cdot 0.99^2}{0.01 \cdot 0.99^2 + 0.99 \cdot 0.01^2}
    = 0.99
  \end{equation*}
\end{example}

\subheader{Статистические рассуждения}

Пусть проверили \(10\,000\) человек. По статистике из них \(100\) больных и
\(9\,900\) здоровых. У \(99\) человек из \(100\) больных анализ окажется
положительным, а у одного~--- отрицательным. У \(99\) из \(9900\) здоровых людей
теста будет положительным, а у остальных~--- отрицательным. Таким образом мы
видим, что даже у человека положительный тест, то вероятность того, что он болен
лишь \(50\)\%.

\begin{example}
  В студии три двери. За одной из них приз. Ведущий предлагает игроку
  угадывать дверь с призом, после чего он открывает одну из оставшихся дверей
  и показывает, что за ней приза нет. Затем он предлагает игроку поменять свой
  выбор. Следует ли игроку соглашаться?

  \solution{} пользоваться формулой Байеса здесь нельзя, т.к. выбор ведущего
  (событие \(A\) в формуле) это не случайное событие. Можно воспользоваться
  примерными статистическими рассуждениями. Пусть игрок сыграл в эту игру
  \(300\) раз. Тогда он \(100\) раз изначально выбрал верную дверь и \(200\)
  раз~--- неверную. Если он после вопроса ведущего поменяет свой выбор, то
  окажется прав в \(200\) случаях из \(300\). Таким образом, если игрок поменяет
  свой выбор, то вероятность его победы будет равна \(\frac{2}{3}\).
\end{example}
