\subsection{%
  Лекция \texttt{23.09.26}.%
}

\subheader{Последовательности независимых испытаний}

\begin{definition}
  Схемой Бернулли называется серия одинаковых независимых испытаний, каждое из
  которых имеет лишь два исхода: произошло интересующее нас событие (успех) или
  не произошло (неудача). Обозначения:

  \begin{enumerate}
  \item
    \(p\)~--- вероятность успеха при одном испытании.

  \item
    \(q = 1 - p\)~--- вероятность неудачи при одном испытании.

  \item
    \(n\)~--- число независимых испытаний.

  \item
    \(Y_n\)~--- число успехов при \(n\) испытаниях.

  \item
    \(N_n\)~--- число неудач при \(n\) испытаниях.

  \item
    Для краткости обозначим \(p(Y_n = k) = p_n(k)\).
  \end{enumerate}
\end{definition}

\subheader{Формула Бернулли}

\begin{theorem}
  Вероятность того, что при \(n\) испытаниях произойдет ровно \(k\) успехов
  будет равна

  \begin{equation*}
    p_n(k) = \comb{n}{k} p^k q^{n - k}
  \end{equation*}
\end{theorem}

\begin{proof}
  Пусть \(A = \set{Y_n = k}\). Рассмотрим один из элементарных исходов,
  благоприятствующих событию \(A\).

  \begin{equation*}
    A_1 = \under{Y \dotsc Y}{k} \under{N \dotsc N}{n - k}
    \qquad
    \prob{Y} = p, \prob{N} = q
  \end{equation*}

  Т.к. испытания независимы, то

  \begin{equation*}
    \prob{A_1} = p^k q^{n - k}
  \end{equation*}

  Остальные элементарные исходы (благоприятствующие \(A\)) отличаются от данного
  лишь расстановкой \(k\) успехов по \(n\) испытаниям, а вероятности их те же
  самые. Всего таких способов расставить будет \(\display{\comb{n}{k}}\). В
  результате получаем искомую формулу.
\end{proof}

\begin{example}
  Вероятность попадания стрелка при одном выстреле равна \(0.8\). Найти
  вероятность того, что из пяти выстрелов три будут точными.

  \solution{} Обозначим \(n = 5\), \(k = 3\), \(p = 0.8\), \(q = 0.2\).
  Подставим это в формулу и получим

  \begin{equation*}
    \probP_5 (3) = \comb{5}{3} \cdot 0.8^3 \cdot 0.2^2 = 0.2048
  \end{equation*}
\end{example}

\subheader{Наиболее вероятное число успехов}

Выясним, при каком значении \(k\), вероятность предшествующего числа успехов \(k
- 1\) будет не более, чем вероятность \(k\) успехов. Формально получаем

\begin{equation*}
  \begin{aligned}
    \probP_n (k - 1) \le \probP_n (k)
  \\
    \comb{n}{k - 1} \cdot p^{k - 1} \cdot q^{n - k + 1}
    \le
    \comb{n}{k} \cdot p^{k} \cdot q^{n}
  \\
    \frac{n!}{(k - 1)! (n - k + 1)!} \cdot q
    \le
    \frac{n!}{k! (n - k)!} \cdot p
  \\
    k \cdot (1 - p) \le (n - k + 1) \cdot p
  \\
    k - k p \le n p - k p + p
  \\
    k \le n p + p
  \end{aligned}
\end{equation*}

Т.к. \(k\) целое, то получаем, что \(n p + p - 1 \le k \le n p + p\). Рассмотрим
3 ситуации.

\subsubheader{Случай I}{\(n p\) целое}

Тогда \(n p + p\)~--- не целое и \(k = n p\) это наиболее вероятное число
успехов.

\subheader{Случай II}{\(n p + p\) не целое}

Тогда \(k = [n p + p]\). Это обобщение случая I.

\subheader{Случай III}{\(n p + p\) целое}

В этом случае получаем два наиболее вероятных числа успехов \(k = n p + p\) и
\(k = n p + p - 1\).

\galleryone{01_04_01}{Геометрическая иллюстрация наиболее вероятного числа
успехов}

Если рассмотреть рисунок \figref{01_04_01} и устремить \(n \to \infty\), то
через полученные точки можно будет провести плавную кривую. Полученный график
(\figref{01_04_02}) называется нормальным распределением. Такое распределение
имеет место быть, если вероятности успеха и неудачи примерно равны. Если же
какая-либо из этих вероятностей много больше другой, то \quote{горб} будет
сильно смещен к одной из границ, при этом одна из его сторон будет крутой, а
другая~--- пологой. Такое распределение называется показательным (об этом мы
поговорим позже).

\galleryone{01_04_02}{Нормальное распределение}

\subheader{Предельные теоремы в схеме Бернулли}

Если требуется найти вероятность точного числа успехов, то применяем следующую
теорему:

\begin{theorem}[Локальная теорема Муавра---Лапласа] \label{thr:local-M-L}
  \begin{equation*}
    \probP_n (k) \Rarr{n \to \infty} \frac{1}{\sqrt{n p q}} \phi(x)
    \qquad
    \begin{cases}
      \phi(x) = \frac{1}{\sqrt{2 \pi}} e^{-\frac{x^2}{2}} \\
      x = \frac{k - np}{\sqrt{n p q}}
    \end{cases}
  \end{equation*}
\end{theorem}

\begin{remark}
  Свойства \(\phi(x)\) (функция Гаусса)

  \begin{enumerate}
  \item
    \(\phi(-x) = \phi(x)\)~--- функция четная.

  \item
    \(x > 5 \implies \phi(x) \approx 0\)~--- функция быстро убывает.
  \end{enumerate}
\end{remark}

Если требуется найти вероятность того, что число успехов находится в данном
диапазоне, то применяем другу теорему:

\begin{theorem}[Интегральная теорема Муавра---Лапласа] \label{thr:int-M-L}
  \begin{equation*}
    \probP_n (k_1 \le k \le k_2) \Rarr{n \to \infty} \Phi(x_2) - \Phi(x_1)
    \qquad
    \begin{cases}
      \Phi(x) = \frac{1}{\sqrt{2 \pi}} \int_0^x e^{-z^2 / 2} \dd z \\
      x_1 = \frac{k_1 - n p}{\sqrt{n p q}} \\
      x_2 = \frac{k_2 - n p}{\sqrt{n p q}}
    \end{cases}
  \end{equation*}
\end{theorem}

\begin{remark}
  Свойства \(\Phi(x)\) (функция Лапласа)

  \begin{enumerate}
  \item
    \(\Phi(-x) = -\Phi(x)\)~--- функция нечетная.

  \item
    \(x > 5 \implies \Phi(x) \approx 0.5\)
  \end{enumerate}
\end{remark}

\begin{remark}
  В различных источниках под функцией Лапласа подразумевают несколько
  иные функции, чаще всего функцию

  \begin{equation*}
    F_0 (x) = \frac{1}{\sqrt{2 \pi}} \int_{-\infty}^x e^{-z^2 / 2} \dd z
  \end{equation*}

  Это функция стандартного нормального распределения.
\end{remark}

\begin{remark}
  Интеграл Пуассона
  
  \begin{equation*}
    \int_{-\infty}^{\infty} e^{-z^2 / 2} \dd z = \sqrt{2 \pi}
  \end{equation*}

  Используя его можно получить, что

  \begin{equation*}
    F_0 (x)
    = \frac{1}{\sqrt{2 \pi}} \int_{-\infty}^0 e^{-z^2 / 2} \dd z
      + \frac{1}{\sqrt{2 \pi}} \int_0^x e^{-z^2 / 2} \dd z
    = \frac{1}{2} + \frac{1}{\sqrt{2 \pi}} \int_0^x e^{-z^2 / 2} \dd z
  \end{equation*}
\end{remark}

\begin{remark}
  Эти формулы (\ref{thr:local-M-L} и \ref{thr:int-M-L}) обычно применяются при
  \(n \ge 100\) и когда \(p\) и \(q\) отличаются друг от друга не слишком
  сильно. Если же \(p\) и \(q\) отличаются друг от друга очень сильно, то
  сходимость все равно будет, но при очень больших \(n\), которые не всегда
  можно получить на практике. Для таких случаев применяется формула Пуассона
  (формула редких событий), которая будет рассмотрена на следующей лекции.
\end{remark}

\begin{example}
  Вероятность попадания стрелка в цель \(0.8\). Сделано \(400\) выстрелов, найти
  вероятность того, что произошло ровно \(330\) попаданий.

  \solution{} Обозначим \(n = 400\), \(p = 0.8\), \(q = 0.2\), \(k = 330\). По
  локальной формуле Лапласа

  \begin{equation*}
    \begin{aligned}
      x
      = \frac{k - n p}{\sqrt{n p q}}
      = \frac{330 - 400 \cdot 0.8}{\sqrt{400 \cdot 0.8 \cdot 0.2}}
      = \frac{330 - 320}{\sqrt{64}}
      = 1.25
    \\
      \probP_{400} (330)
      \approx \frac{1}{\sqrt{n p q}} \phi(x)
      = \frac{1}{8} \phi(1.25)
      = 0.125 \cdot 0.1826
      \approx 0.0228
    \end{aligned}
  \end{equation*}
\end{example}

\begin{example}
  Вероятность попадания стрелка в цель \(0.8\). Сделано \(400\) выстрелов, найти
  вероятность того, что произошло от \(312\) до \(336\) попаданий.

  \solution{} Обозначим \(n = 400\), \(p = 0.8\), \(q = 0.2\), \(k_1 = 312\),
  \(k_2 = 336\). По интегральной формуле Лапласа
  
  \begin{equation*}
    \begin{aligned}
      x_1
      = \frac{k_1 - n p}{\sqrt{n p q}}
      = \frac{312 - 400 \cdot 0.8}{\sqrt{400 \cdot 0.8 \cdot 0.2}}
      = \frac{312 - 320}{\sqrt{64}}
      = -1
        \qquad
      x_2
      = \frac{k_2 - n p}{\sqrt{n p q}}
      = \frac{336 - 400 \cdot 0.8}{\sqrt{400 \cdot 0.8 \cdot 0.2}}
      = \frac{336 - 320}{\sqrt{64}}
      = 2
    \\
      \probP_{400} (312 \le k \le 336)
      \approx \Phi(2) - \Phi(-1)
      = \Phi(2) + \Phi(1)
      = 0.4772 + 0.3413
      \approx 0.8185
    \end{aligned}
  \end{equation*}
\end{example}

Вернемся к статистическому определению вероятности. Пусть проводится \(n\)
независимых испытаний, событие \(A\) появилось \(n_A\) раз. Относительная
частота равна \(\frac{n_A}{n}\), тогда статистическое определение вероятности
утверждает, что \(\frac{n_A}{n} \approx \prob{A} = p\).

\subheader{Вероятность отклонения относительной частоты от вероятности события}

Пусть \(p\)~--- вероятность события \(A\), а \(\frac{n_A}{n}\)~--- относительная
частота. По интегральной формуле Лапласа

\begin{equation*}
  \begin{aligned}
    \prob{\abs{\frac{n_A}{n} - p} < \epsilon}
    = \prob{-n \epsilon < n_A - n p < n \epsilon} 
    = \prob{n p - n \epsilon < n_A < n p + n \epsilon} 
  \\
    x_1
    = \frac{n p - n \epsilon - n p}{\sqrt{n p q}}
    = -\frac{n \epsilon}{\sqrt{n p q}}
    \qquad
    x_2
    = \frac{n p + n \epsilon - n p}{\sqrt{n p q}}
  \\
    \prob{\abs{\frac{n_A}{n} - p} < \epsilon}
    \approx \Phi \prh{\frac{n \epsilon}{\sqrt{n p q}}}
      - \Phi \prh{-\frac{n \epsilon}{\sqrt{n p q}}}
    = \Phi \prh{\frac{n \epsilon}{\sqrt{n p q}}}
      + \Phi \prh{\frac{n \epsilon}{\sqrt{n p q}}}
    = 2 \Phi \prh{\frac{\sqrt{n} \epsilon}{\sqrt{p q}}}
  \end{aligned}
\end{equation*}

Итак, получили следующую формулу

\begin{equation*}
  \prob{\abs{\frac{n_A}{n} - p} < \epsilon}
  \approx 2 \Phi \prh{\frac{\epsilon}{\sqrt{p q}} \sqrt{n}}
\end{equation*}

\subheader{Закон больших чисел Бернулли}

Рассмотрим выведенную формулу. При \(n \to \infty\) аргумент функции Лапласа
будет стремиться к \(\infty\), а значит по \(2^{\circ}\) свойству значение
функции будет стремиться к \(\frac{1}{2}\). Значит

\begin{equation*}
  \lim_{n \to \infty} \prob{\abs{\frac{n_A}{n} - p} < \epsilon} = 1
\end{equation*}

Это равенство и называется законом больших чисел Бернулли. Такая сходимость
называется сходимостью по вероятности.

\begin{example}
  Теща Кисы Воробьянинова с вероятностью \(0.9\) зашила бриллианты в одном из
  \(12\) стульев, а с вероятностью \(0.1\) подшутила над ним. В первых
  одиннадцати стульях бриллиантов не оказалось. Какова вероятность того, что они
  найдутся в двенадцатом стуле?

  \solution{} \(A_i\)~--- бриллианты в \(i\)-том стуле, \(\prob{A_i} =
  \frac{3}{40}\). Получаем

  \begin{equation*}
    \begin{aligned}
      \condprob{A_{12}}{\bar{A_1} \dotsc \bar{A_{11}}}
      = \frac{\prob{A_{12} \bar{A_1} \dotsc \bar{A_{11}}}}
        {\prob{\bar{A_1} \dotsc \bar{A_{11}}}}
    \\
      \prob{\bar{A_1} \dotsc \bar{A_{11}}}
      = 1 - \prob{A_1 + \dotsc + A_{11}}
      = 1 - \prh{\prob{A_1} + \dotsc + \prob{A_{11}}}
      = \frac{7}{40}
    \\
      \prob{A_{12} \bar{A_1} \dotsc \bar{A_{11}}}
      = \prob{A_{12}}
      = \frac{3}{40}
    \\
      \condprob{A_{12}}{\bar{A_1} \dotsc \bar{A_{11}}}
      = \frac{3}{7}
    \end{aligned}
  \end{equation*}
\end{example}

\begin{example}
  Для оценки доли \(p\) курящих людей берется выборка объема \(n\). Далее
  делается оценка \(p^* = \frac{n_k}{n}\). Какой должен быть объем \(n\), чтобы
  с вероятностью \(\gamma = 0.95\) данная оценка отличалась от истинного
  значения не более чем на \(\epsilon = 0.01\).

  \solution{} используем формулу отклонения относительной частоты от
  вероятности

  \begin{equation*}
    \begin{aligned}
      \prob{\abs{p^* - p} \le \epsilon}
      \approx 2 \Phi \prh{\frac{\sqrt{n} \epsilon}{\sqrt{p q}}}
      \ge 0.95
    \\
      \Phi \prh{\frac{\sqrt{n} \epsilon}{\sqrt{p q}}} \ge 0.475
    \\
      \frac{\sqrt{n} \epsilon}{\sqrt{p q}} \ge 1.96
    \\
      \sqrt{n} \ge \frac{1.96 \cdot \sqrt{p q}}{\epsilon}
    \\
      n \ge \frac{1.96^2 \cdot p q}{0.01^2}
    \\
      n \ge 196^2 \cdot p (1 - p)
    \end{aligned}
  \end{equation*}

  Функция \(p (1 - p)\) достигает наибольшего значения при \(p = 0.5\), значит
  получаем \(n \ge 9604\).
\end{example}
